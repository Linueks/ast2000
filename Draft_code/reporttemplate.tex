\documentclass[a4paper,10pt]{article}
\usepackage[utf8]{inputenc}
\usepackage{enumerate}
\usepackage{amsmath}
\usepackage{graphicx}
\usepackage{listings}

\title{Oppskrift på rapport}
\author{Håvard Tveit Ihle}
\date{}

\begin{document}
\maketitle

\section{Introduksjon}

Hva er problemstillingen du skal løse? Beskriv problemet kort, og gi en oversikt over hele rapporten. 

\section{Metode/Fremgangsmåte}

Her forklarer du hvordan du gikk fram for å løse oppgaven, hvilke valg du tok og hvilke metoder du brukte. 

Hvis du har noen analytiske utregninger/mellomregninger, da hører de oftest hjemme i denne delen. 

\subsection{Tips:}

Her kan du skrive ligninger:
\begin{equation}
 E = mc^2,
\end{equation}
eller for eksempel en utledning (rekke av ligninger):
\begin{align}
  I &= \int_0^\pi \sin(x) dx \\ 
    &= -\cos(x)\Big|_0^\pi\\
    &= 2.
\end{align}

Hvis du har lyst til å liste opp punkter kan du bruke:
\begin{itemize}
\item Første punkt!
\item Andre punkt!
\end{itemize}
eller
\begin{enumerate}
\item Første punkt!
\item Andre punkt!
\end{enumerate}

Du kan indikere en vektor med $\vec a$ eller ${\bf a}$. Matriser kan du også ta med:
\begin{equation}
 \left( \begin{array}{cc}
a_{11} & a_{12} \\
a_{21} & a_{22} \end{array} \right)
\end{equation}


\section{Resultater}

Her beskriver du kvantitivt de resultatene du har oppnådd. Gjerne i form av en eller flere figurer/plot eller tabeller. 

\subsection{Tips:}
Hvis du vil legge til en figur kan du gjøre dette: 
% % (kommenter inn og legg filen ''eksempel.pdf'' i samme mappen som denne .tex - filen)
% \begin{figure}[!h]
% \centering
% \includegraphics[width=12cm]{eksempel.pdf}
% \caption{Dette er et awesome plot!}
% \label{myPlot}
% \end{figure}

Husk å alltid referere til alle figurene i teksten ved å bruke labels: %Se figur \ref{myPlot}.

Hvis du har et plot, så husk å gi labels til aksene (med enheter!). Vær også veldig tydelig på hvilke parametre som er brukt i plottet, all slik informasjon burde være tilgjengelig i plottet eller i figurteksten. 

Du kan også legge til tabeller (for eksempel tabell \ref{NeonTable}):
\begin{table}[ht]
\begin{tabular}{ c | c | c | c | c |c }
  $\alpha$ & $\beta$ & $E_{VMC}$ & $\sigma$ & $E_0$& N  \\
\hline
  10.22 & 0.0914   & -127.9(1) & 0.012 & -128.94 & $1.26\cdotp10^8$\\
\end{tabular}
\caption{Table of our results for the Neon atom. $\alpha$ and $\beta$ are the optimal variational parameters. $E_{VMC}$ gives our best estimate of the ground state energy, with a standard deviation of $\sigma$. $E_0$ is the estimated exact energy. $N$ is the number of MC-samples.\label{NeonTable}}
\end{table}

\section{Diskusjon og konklusjon}

Prøv å sette resultatene dine i perspektiv. Er resultatene rimelige? Var det noe som var uventet? 

Oppsummer resultatene og prosjektet i sin helhet. Prøv å trekke konklusjoner fra resultatene og diskusjonen din. 

% For spesielt interesserte kan dere lage referanseliste (hvis du har brukt kildene \cite{Shankar} og \cite{Griffiths}), men dette er veldig overkill i dette faget. 
% \begin{thebibliography}{9}
% \bibitem{Shankar}
% Shankar, R (1994). \emph{Principles of Quantum Mechanics.} 2nd edition.
% \bibitem{Griffiths}
% Griffiths, D. (2005). \emph{Introduction to Quantum Mechanics.} 2nd edition.
% \end{thebibliography}

\appendix 
\section{Vedlegg}

Hvis du har noen vedlegg, så kan du legge til det. Hvis du for eksempel har lyst til å legge ved koden din i rappoten, så hører den hjemme her. 

\end{document}
